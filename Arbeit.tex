% Dokumentenart und Schriftgröße
\documentclass[
    12pt,
    a4paper
]{article}

% Verfügbarmachen externer Funktionalitäten
\usepackage[
    left=4cm,
    right=1.5cm,
    top=3cm,
    bottom=2cm
]{geometry}
\usepackage[
    toc,
    style=alttree,
    acronym,
    nopostdot,
    nonumberlist
]{glossaries}
\usepackage[ngerman]{babel}
\usepackage[export]{adjustbox}
\usepackage[titles]{tocloft}
\usepackage[
    backend=biber,
    style=numeric
]{biblatex}
\usepackage[T1]{fontenc}
\usepackage{verbatim}
\usepackage{minted}
\usepackage{hyperref}
\usepackage{setspace}
\usepackage{graphicx}
\usepackage{ragged2e}
\usepackage{fontspec}
\usepackage{fancyhdr}
\usepackage{svg}
\usepackage{tocbibind}
\usepackage{float}
\usepackage{tikz}
\usepackage{tabularx}

% Variableneinbindung
% Hier erfolgt die Eintragung der Metadaten (z. B. Name des Students).

\newcommand{\Phasentitel}{Projektstudium im SS 23}
\newcommand{\Thema}{Thema der Praxisphase}
\newcommand{\NameStudent}{Max Mustermann}
\newcommand{\StraßeStudent}{Musterstraße 1}
\newcommand{\OrtStudent}{12345 Musterstadt}
\newcommand{\Matrikelnummer}{123456789}
\newcommand{\Hochschulbetreuer}{Prof. Dr. Hans Meier}
\newcommand{\Fachbetreuer}{Herr Frank Müller}
% Ist man dualer Student bei einem Unternehmen oder einer Behörde? (Relevant für das Deckblatt)
\newcommand{\UnternehmenOderBehörde}{Unternehmen}
\newcommand{\Unternehmen}{Musterfirma GmbH}
\newcommand{\DatumEinreichung}{01.01.2023}

% Datei, welche die Einträge für das Literaturverzeichnis enthält
\addbibresource{Literatur.bib}

% Pfad zu Bildern setzen
\graphicspath{{Bilder/}}

% Abkürzungsverzeichnis
\makeglossaries

% Einrückungen im Abbildungsverzeichnis verhindern
\renewcommand\cftfigindent{0pt}

% Seitenzahl in oberer rechter Ecke einblenden
\pagestyle{fancy}
\fancyhf{}
\fancyhead[R]{\thepage}
\renewcommand{\headrulewidth}{0pt}

% Schriftart des Dokuments
\setmainfont{Arial}

% Zeilenabstand
\setstretch{1.5}

% Design des Syntax-Highlightings
\usemintedstyle{staroffice}

% Inhaltsverzeichnis ohne Punkte zwischen Überschrift und Seitenzahl
\renewcommand{\cftdot}{}

% Farbe von Links und Titel des Dokuments
\hypersetup{
    colorlinks=true,
    citecolor=black,
    linkcolor=black,
    urlcolor=blue,
    pdftitle={\Thema{}}
}

% Liste verwendeter Abkürzungen
% Hier erfolgt die Eintragung in der Arbeit verwendeter Abkürzungen.

\newacronym{zb}{z. B.}{zum Beispiel}

\begin{document}

% Seitennummerierung deaktivieren
\pagenumbering{gobble}

% Logo-Bereich
\noindent
\begin{minipage}[t]{\textwidth}
    \raisebox{-0.28cm}{\includesvg[height=1.7cm]{THM-Logo}}
    \hfill
    \includesvg[height=1.7cm]{StudiumPlus-Logo}
\end{minipage}\\[.5cm]

\begin{Center}
    \textbf{\Phasentitel{}\\[1cm]}
\end{Center}

\noindent
Thema:\\
\textbf{\Thema{}\\[1.5cm]}

% Angaben zum Studenten und des Unternehmens
\begin{FlushLeft}
    \begin{tabularx}{\textwidth}{@{}l@{\hspace{2.5cm}}X@{}}
        Vorgelegt von:             & \NameStudent{}       \\
                                   & \StraßeStudent{}     \\
                                   & \OrtStudent{}        \\
        Matrikelnummer:            & \Matrikelnummer{}    \\[1.5cm]
        Eingereicht bei                                   \\
        Hochschulbetreuer:         & \Hochschulbetreuer{} \\
        Fachbetreuer:              & \Fachbetreuer{}      \\[1.5cm]
        \UnternehmenOderBehörde{}: & \Unternehmen{}       \\
        Eingereicht am:            & \DatumEinreichung{}
    \end{tabularx}
\end{FlushLeft}

\clearpage

% Seitennummerierung mit großen römischen Zahlen
\pagenumbering{Roman}

% Zählvariable für die mit römischen Zahlen nummerierten Seiten
\newcounter{Römisch}
\setcounter{Römisch}{1}

% Inhaltsverzeichnis
\tableofcontents

\stepcounter{Römisch}

\clearpage

% Angaben zum Abkürzungsverzeichnis
\glsfindwidesttoplevelname
\printglossary[
    type=\acronymtype,
    title=Abkürzungsverzeichnis,
    toctitle=Abkürzungsverzeichnis
]

\stepcounter{Römisch}

\clearpage

% Abbildungsverzeichnis
\listoffigures

\stepcounter{Römisch}

\clearpage

% Tabellenverzeichnis
\listoftables

\stepcounter{Römisch}

\clearpage

% Seitennummerierung mit arabischen Zahlen
\pagenumbering{arabic}

% Einfügen der Kapitel des Textteils
\foreach \i in {01,02,...,99} {
        \edef\Datei{Textteil/Kapitel-\i}

        \IfFileExists{\Datei}{
            \input{\Datei}

            \clearpage
        }{}
    }

\pagenumbering{Roman}

% Seitenzahl festlegen
\setcounter{page}{\value{Römisch}}

% Literaturverzeichnis
\printbibliography[
    title=Literaturverzeichnis,
    heading=bibintoc
]

\end{document}